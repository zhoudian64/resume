% !TEX TS-program = xelatex
% !TEX encoding = UTF-8 Unicode
% !Mode:: "TeX:UTF-8"

\documentclass{resume}
\usepackage{zh_CN-Adobefonts_external} % Simplified Chinese Support using external fonts (./fonts/zh_CN-Adobe/)
%\usepackage{zh_CN-Adobefonts_internal} % Simplified Chinese Support using system fonts
\usepackage{linespacing_fix} % disable extra space before next section
\usepackage{cite}

\begin{document}
\pagenumbering{gobble} % suppress displaying page number

\name{周典}

\basicInfo{
  \email{zhoudian64@gmail.com} \textperiodcentered\ 
  \phone{(+86) 180-160-99795} \textperiodcentered\ 
  \linkedin[Dian Zhou]{https://www.linkedin.com/in/dian-zhou-8011041a3/}}

\section{\faGraduationCap\  教育背景}
\datedsubsection{\textbf{上海大学}, 上海}{2017 -- 至今}
\textit{在读本科生}\ 计算机科学与技术

\section{\faUsers\ 项目经历}
\datedsubsection{\textbf{SHUHelper, 上海大学熟知网手机版}}{2019年3月 -- 至今}
\role{微服务, Vue, K8S}{个人项目与学校官方项目}
\begin{onehalfspacing}
将学生代理登录写成代理服务后,完成了比学校更易用的手机端校园各类信息查询和请求 \newline
https://github.com/shuosc/SHUHelper \&
https://github.com/shuosc/shuZhiNet
\begin{itemize}
  \item Go实现对上海大学熟知网的爬虫和格式化后的Web服务, Python实现了第一版的对图书馆的爬虫
  \item Vue.js的单页前端
  \item 容器化部分后端服务和配置Nginx对来自K8S的服务的多前端反向代理
  \item 在阿里云的多台学生机上搭建K3S
  \item 部分后端服务的travis CICD
\end{itemize}
\end{onehalfspacing}

\datedsubsection{\textbf{上海大学手语及聋人研究中心}}{2019年6月 -- 至今}
\role{Golang, Vue}{学校官方项目}
上海大学手语的网上教学平台 https://github.com/AdrianDuan/CCSL
\begin{itemize}
  \item 实现评论区的后端接口
  \item 研究团队人员介绍的前端页面
  \item 部署至https://ccsl.shu.edu.cn/
\end{itemize}

\datedsubsection{\textbf{ICPC 2019} 上海}{2019 年11月}
\role{Ubuntu Desktop}{学校官方运维}
\begin{onehalfspacing}
\begin{itemize}
  \item pssh批量配置开发环境
  \item 无密码登录、远程控制等GNOME桌面安全管控
\end{itemize}
\end{onehalfspacing}

\datedsubsection{\textbf{RISC-V32i不完全指令集计算机}}{2019年8月}
\role{verilog}{个人项目}
一个实现了除了中断以外的无流水线计算机 https://github.com/zhoudian64/hardware
\begin{itemize}
  \item CPU部分包含了从内存取指,指令译码器,ALU执行
  \item 经过4位采样的8位FIFO PS/2键盘
  \item 640*480分辨率3至24位颜色深度VGA显示输出和80*60的字符输出方式
  \item 2进制的显示键盘输入程序
\end{itemize}

\datedsubsection{\textbf{Nintendo Switch自动化手柄}}{2020年3月}
\role{C}{个人项目}
一个运行在cortex-M3 MCU上的自动化USB手柄,可以在Nintendo Switch上按用户编写的C结构体数组脚本自动化完成手柄任务
\begin{itemize}
  \item 使用STM32CubeMX初始生成了FreeRTOS和USBD等HAL库,由于STM的代码的LICENSE分布在各个文件,所以项目尚未开源
  \item 使用C编写了指示灯线程,主要USB发包线程和脚本的程序计数器线程,提供4位7段二极管任务来DEBUG脚本
\end{itemize}

% Reference Test
%\datedsubsection{\textbf{Paper Title\cite{zaharia2012resilient}}}{May. 2015}
%An xxx optimized for xxx\cite{verma2015large}
%\begin{itemize}
%  \item main contribution
%\end{itemize}

\section{\faCogs\ 编程语言}
% increase linespacing [parsep=0.5ex]
\begin{itemize}[parsep=0.5ex]
  \item 了解: shell, 包含不限于{python, go, rust, node.js}标准库/轻量后端框架WEB, Vue.js
  \item 接触过: C (cortex-M Bare Metal; glibc; openmp), C++ (CUDA), verilog, MATLAB, MIPS/RISC-V32i ASM
\end{itemize}

\section{\faCogs\ 系统、数据库及工具}
\begin{itemize}[parsep=0.5ex]
    \item 了解: git, docker
    \item 接触过: Linux, postgresql, kubernetes, tidb-operator,Makefile, Nginx, github tools, FreeRTOS
\end{itemize}{}

\section{\faHeartO\ 获奖情况}
\datedline{\textit{市二等奖}, 全国大学生数学建模竞赛}{2018 年9 月}

\section{\faInfo\ 其他}
% increase linespacing [parsep=0.5ex]
\begin{itemize}[parsep=0.5ex]
  \item GitHub: https://github.com/zhoudian64
  \item https://github.com/shuosc/ 当前负责人
  \item 语言: 英语 - CET4
\end{itemize}

%% Reference
%\newpage
%\bibliographystyle{IEEETran}
%\bibliography{mycite}
\end{document}
